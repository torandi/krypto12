\documentclass[a4paper,11pt]{article}

\usepackage[english]{babel}
\usepackage{parskip}

\usepackage{rapportfram}
\author{Andreas Tarandi\\890416-0317}
\title{Homework I}
\titleextra{No study group}
\course{DD2448}

\begin{document}
	\maketitle

	\section{}
		gcd(a, b) is all the common factors in a and b.
		\begin{itemize}
			\item The $b > a$ check ensures that a always is at least as big as b. 
			\item b == 0 check returns a ( gcd(n, 0) = 0)
			\item If b and a are both even, then 2 is a factor in both and thus also a factor in the gcd.
			\item If one of a or b but not both is even then 2 is not a factor in the gcd and the even operand may be divided with 2
			\item This is almost the same step as in the Euclidean algorithm, except that we only remove one b at a time (in the euclidean algorithm floor(a/b) b:s are removed each iteration. \\
				Taking gcb(a,b) = gcb(b, a-b) is valid since the difference between a and b must be a factor of the gcd.
		\end{itemize}

		The number of recursive calls are at most the number of factors in the biggest operand.
		
	\section{}
		To find X we use the chinese remainder theorem. We have the system \\
		 $ X \equiv a_1 mod N $\\
		 $ X \equiv a_2 mod (N+1) $\\
		
		Which gives us $ X \equiv a_{1}U_1 + a_{2}U_2 mod N(N+1)$ (from the chinese remainder therem). From Euclid's extended algorithm we get $1 = aN + b(N-1) $ which in this case is trivial, $a = -1$, $b = 1$. Where $U_1 = a*N$, $U_2 = b*(N-1)$.

		This gives us the result:\\
		$U_1 = -8904160317$\\
		$U_2 = 8904160318$\\
		$X \equiv 123456789*-8904160317 + 987654321*8904160318$ $mod$ $N(N-1)$\\
		$X \equiv 7694953371471391965 $ $ mod$ $N(N-1)$\\

		That is X = 7694953371471391965 (both $a_1$ and $a_2$ are less than N and (N+1)


	\section{}
		For this radix sort would be a good choice:

		\begin{verbatim}
         def sort(list_num) 
            current_divisor = 1
            (0..Math.log(list.max)).each do |digit_num|
               sublists = [[], [], [], [], [], [], [], [], [], []]
               list.each do |num| # Iterate through, in order
                  digit = ( num / current_divisor ) % 10
                  # append to sublist (in order of appearance)
                  sublists[digit] << num 
               end
# Merge lists:
               list = sublists.flatten
               # List is now sorted up to digit digit_num
               current_divisor*=10 # Increase divisor
            end

            list #Return list
         end
		\end{verbatim}

		This will work since we sort be each digit and keep the order from the previous digits.

		This algorithm has the complexity O(n*k) where k is $log_{10} $ list.max. In this case the the magnitude of the numbers are $n^{10}$, which gives us O($n*log_{10}(n^{10}) = n*log_{10}n * 10$) = O($n * log_{10} n$) worth noticing here is that it is $log_{10}$, and not $log_2$ which is normaly intended when talking complexity. $log_{10}(n)$ is negligible since it's much smaller than n (ex $ n = 1 000 000 000 $ gives $ log_{10}(n) = 9$ The algorithm can therefore be concidered to run in linear time.

	\section{}
		Here is a nice quick and dirty solution:\\
		\begin{verbatim}
      std::vector<int> sort(std::vector<int> list) {
         std::vector<int> out;
         std::map<int, int> set;
         std::vector<int>::iterator it;
         std::map<int,int>::iterator it2;
         for(it = list.begin(); it != list.end(); ++it) {
            ++set[*it];
         }
         for(it2 = set.begin(); it != set.end(); ++it) {
            for(int i = 0; i < it2->second; ++i) {
               out.push_back(it2->first);
            }
         }
         return out;
      }
		\end{verbatim}

		This is probably not what you expected here, but it does the job in the required time. We have two for-loops over all elements O(2n) = O(n) and the insertion (and the lookup if the element already exists, done in the same call) into the map takes O(log [number of elements in the map]) which is at most O(log(m)), this gives us O(n*log(m)) which is the required complexity.

\end{document}
