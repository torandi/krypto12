\documentclass[a4paper,11pt]{article}

\usepackage{listings}
\usepackage[T1]{fontenc}
\usepackage[swedish, english]{babel}
\usepackage{parskip}

\usepackage{rapportfram}

\author{Andreas Tarandi\\890416-0317}
\title{Homework I}
\titleextra{Ingen studiegrupp}
\course{DD2448}

\begin{document}
	\maketitle

	\section{}
		\subsection{Program}
			De programmen som jag har skrivit för denna uppgift är följande (i bokstavsordning). Samtliga ruby program är skrivna för ruby 1.9.2.
			\begin{itemize}
				\item \textbf{coincidence.rb}	\\
					Beräknar index of coincidence i given fil med varierande nyckellängd och kan även per rad testa alla kombinationer av nycklar per tecken i nyckeln och därmed beräkna hur nära texten i raden ligger engelska. Genom detta kan den äver föreslå en nyckel.\\
					Använder filen ''english.rb'' som innehåller sannolikhetsfördelningen av tecken i vanlig engelska.
					\\
					Kräver gemet ''colorize''	
				\item \textbf{decrypt.rb}	\\
					Avkrypterar given fil med given nyckel (av längd n)
				\item \textbf{freq.cpp}\\
					Utför frekvensanalys på given fil och beräknar statistik och ritar histogram
				\item \textbf{hex\_to\_ascii.rb}\\
					Tar par av nummer från indata och tolkar som ascii och testar därefter att skifta texten med alla nummer i en given mängd. 
				\item \textbf{kasiki.cpp}	\\
					Räknar avstånd mellan återkommande textsegment av längd tre och sparar avstånd och antal förkomster av avståndet.
			\end{itemize}
		\subsection{Skiffer A}
			\subsubsection{Metod}
			Det första steget var att utföra frekvensanalys på skiffertexten. För detta använde jag programmet freq.cpp för att 
			ta fram både siffror och histogram på frekvensen av tecken, bigram och trigram.

			Vid en första anblick av resultatet av frekvensanalysen ser man att W är klart vanligast. Dessutom ser man på fördelningen över tecknena att det tycks vara ett monoalfabetiskt substitutionsskiffer..
			
			E är visserligen den vanligaste bokstaven i engelska språket, men vid en analys av hur W är utplacerat i texten ser man snabbt att W är mellanrum. 

			Jag ersatte därmed samtliga W i texten med mellanrum och körde frekvensanalysen ytterligare en gång. Härefter studerade jag frekvensen av tecken, bigram och trigram och pusslade mig fram till klartexten genom att hitta skiffer-klartext-kopplingen bokstav för bokstav, givetvis med start i de vanligaste tecknena och slut i de ovanligaste.

			\subsubsection{Klartext}
				\lstinputlisting[breaklines]{krypto12Aplaintext.txt}

\end{document}
