\documentclass[a4paper,11pt]{article}

\usepackage{listings}
\usepackage[T1]{fontenc}
\usepackage[english, english]{babel}
\usepackage{parskip}

\usepackage{rapportfram}

\author{Andreas Tarandi\\890416-0317}
\title{Homework B}
\titleextra{Study group: Meidi Tõnisson, Lovisa Skeppe}
\course{DD2448}

\def\leg[#1]#2{\left(\frac{#1}{#2}\right) }

\begin{document}
	\maketitle

	\section*{1. CS = (Gen, Enc, Dec)}
		(1) Sematical security is defined as the infeasability of a computational bound adversary to find any significant information about the message given the cryptotext and public key. \\
		(2) This includes the attack where the adversary generates ciphertexts using random plaintext messages and the public key.
	
	\subsection*{1a. Semantical security of $CS^2$}
		$CS^2$ could be seen as simply running $CS$ multiple times, since that is what it actually does. 
		Let the complexity of getting any information from $CS$ be $O(p)$ where p is a polynom and $O(p)$ is above what can possibly be caluculated (that is, $CS$ is sematically secure). 
		But then the complexity of $CS^2$ should be $O(2p) = O(p)$. This assumes that it doesn't get easier to get any information from a crypto system if one is given multiple ciphertexts encrypted with the same public key.
		But (2) gives us that $CS$ is secure, even under these circumstances, so therefore $CS^2$ is sematically secure.

	\subsection*{1b. Semantical security of $CS^{k(n)}$}
		Using the same method as in 1a it is not hard to see that $CS^{k(n)}$ is sematical 
		secure if $O(k(n)*p) \ge O(p)$, which boils down to $k(n) \ge 1$. 
		But for numbers less than 1 the definition of $CS^k$ doesn't make any sense anyway, so $CS^{k(n)}$ is secure for any polynom k(n).


	\section*{3.}
	$\leg[a]{p}$


\end{document}
