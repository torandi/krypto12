\documentclass[a4paper,11pt]{article}

\usepackage{listings}
\usepackage[T1]{fontenc}
\usepackage[english, english]{babel}
\usepackage{parskip}

\usepackage{rapportfram}

\author{Andreas Tarandi\\890416-0317}
\title{Homework C}
\titleextra{Study group: Peter Boström}
\course{DD2448}

\begin{document}
	\maketitle

	\section*{1. }
		\subsection*{a) DH implies DL}
			Assume there exists an adversary $A_{DL}$ such that
			$ A_{DL}(g, y) = log_g{y} ] $, that is that the DL-assumption does not hold.
		
			If we choose $g^x$ as $y$ then $A_{DL}(g, g^x) = log_g{g^x} = x$

			But then we can write the DH-adversary as $A_{DH}(g^a, g^b) = g^{ab} = g^{a log_g{g^b}} = g^{a * A_{DL}(g, g^b)} $, that is as a function of the DL-adversary. 

			But then since we in (1) assumed that $A_{DL}$ exists, $A_{DH}$ will also exist.

			This means that if the DL-implication does not hold, then neither does the DH-implication: $\neg DL \to \neg DH $. 

			We can then apply the rule of transposition ($\neg P \to \neg Q \vdash Q \to P$) and we have $DH \to DL$ \\
			\\
			Q.E.D.
			
	\subsection*{b) DDH implies DH }
		
	Assume there exists an adversary $A_{DH}$ such that $ A_{DH}(g^a, g^b) = g^{ab}$, that is, the DH-assumption does not hold. (2)

	Then if we create an DDH-adversary $A_{DDH}(a, b, c}) = (c == A_{DH}{a, b)$,
	we can distinguish between the probability distribution for $Pr [ A_{DDH}(g^a, g^b, g^{ab}) = 1] $ from the one for $ Pr [ A_{DDH}(g^a, g^b, g^c) = 1 ] $, meaning that DDH-does not hold.

	We have that $\neg DH \to \neg DDH$, and by one again applying the rule of transposition we get $DDH \to DH$. \\
	\\
	Q.E.D.

		 
		

\end{document}
