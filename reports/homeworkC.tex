\documentclass[a4paper,11pt]{article}

\usepackage{listings}
\usepackage[T1]{fontenc}
\usepackage[english, english]{babel}
\usepackage{parskip}

\usepackage{rapportfram}

\author{Andreas Tarandi\\890416-0317}
\title{Homework C}
\titleextra{Study group: Peter Boström and Meidi Tönisson}
\course{DD2448}

\begin{document}
	\maketitle

	\section*{1. }
		\subsection*{a) DH implies DL}
			Assume there exists an adversary $A_{DL}$ such that
			$ A_{DL}(g, y) = log_g{y} $, that is that the DL-assumption does not hold.
		
			If we choose $g^x$ as $y$ then $A_{DL}(g, g^x) = log_g{g^x} = x$

			But then we can write the DH-adversary as $A_{DH}(g^a, g^b) = g^{ab} = g^{a log_g{g^b}} = g^{a * A_{DL}(g, g^b)} $, that is as a function of the DL-adversary. 

			But then since we in (1) assumed that $A_{DL}$ exists, $A_{DH}$ will also exist.

			This means that if the DL-implication does not hold, then neither does the DH-implication: $\neg DL \to \neg DH $. 

			We can then apply the rule of transposition ($\neg P \to \neg Q \vdash Q \to P$) and we have $DH \to DL$ \\
			\\
			Q.E.D.
			
	\subsection*{b) DDH implies DH }
		
	Assume there exists an adversary $A_{DH}$ such that $ A_{DH}(g^a, g^b) = g^{ab}$, that is, the DH-assumption does not hold. (2)

	Then if we create an DDH-adversary $A_{DDH} (a, b, c) = (c == A_{DH} (a, b)$,
	we can distinguish between the probability distribution for $Pr [ A_{DDH}(g^a, g^b, g^{ab}) = 1] $ from the one for $ Pr [ A_{DDH}(g^a, g^b, g^c) = 1 ] $, meaning that DDH-does not hold.

	We have that $\neg DH \to \neg DDH$, and by one again applying the rule of transposition we get $DDH \to DH$. \\
	\\
	Q.E.D.

	\section*{2. }

	\subsection*{a)}

	$ \psi(g^a, g^b) = \psi(g, g)^{ab}$ (1)\\
	$ \psi(g^1, g^c) = \psi(g, g)^{1*c}$ (2)\\
	Since we know that $\psi(g, g) \neq 1$ we can just compare the result from (1) and (2) to determine the outcome. 
	Therefore the Diffie-Hellman decision problem is when using the map.

	\subsection*{b) }
	In a generalized three party Diffie-Hellman key exchange using the map each part would 
	start by sending $g$ raised to his exponent to the two other parts.

	Each part will then have his own exponent and $g$ raised to the two others separatly.
	If he then inputs the two values from the other parts into the map an raises the output
	to his own exponend he will have the secret key.

	Example: (A's calculations):\\
	$ \psi(g^b, g^c)^a = \psi(g, g)^{bca} = \psi(g, g)^{abc} $

	\subsection*{c) }
	The decision Diffie-Hellman problem under which our scheme would be secure is as follows:
	In a group $G$ of prime order $p$ with generator $g$ and $a, b, c, d \in Z_q$ randomly chosen then for every polynomial time algorithm $A$\\
	$|Pr [ A(g^a, g^b, g^c, g^{abc} = 1 ] - Pr [ A(g^a, g^b, g^c, g^d) = 1] | $ is negligible.

	\subsection*{d) }
	Assume that a man-in-the-middle intercepts all traffic between the parts. He would have a
	own private exponent and have a separate key for each channel to the parts. For each channel
	he recives a public value ($g^x$, $x \in {a, b}$) and then sends one of the other parts public value ($g^y$, $y \in {a,b}$, $y \neq x$) and his own public value($g^z$). 
	The key for that channel then becomes $\psi(g^x, g^y)^z$. Using this techique all the parts
	will be under the assumption that they communicate with each other while they in fact are
	communicating through a fourth part.

	One thing that might tell the parts that something is wrong is if all the communication 
	comes from the same IP-address, so the man-in-the-middle would probably have to use two
	separatly adresses.




		 
		

\end{document}
