\documentclass[a4paper,11pt]{article}

\usepackage{listings}
\usepackage[T1]{fontenc}
\usepackage[english, english]{babel}
\usepackage{parskip}

\usepackage{rapportfram}

\author{Andreas Tarandi\\890416-0317}
\title{Homework D}
\titleextra{Study group: }
\course{DD2448}

\begin{document}
	\maketitle

	\section*{1. }
		To be collision resistant the following must hold:\\
		For all polynomial time adveraries A:
		$P[A(g, N) = (x, x') \wedge x \neq x' \wedge g^x mod N = g^{x'} mod N]$ is negligible\\
		That is, there exist no A that can find $x$ and $x'$ that are not equal for which\\
		$g^x \equiv g^{x'} mod N$\\
		which gives $x' = x + ord(g)k$, $k \in Z$.\\
		So finding a collision becomes as hard as finding the order of g, so if we can make the assumption that finding ord(g) of an element $g \in pq$ where p and q are safe primes then this proof would be done.

		There are basicaly two ways of finding the order of an element $g \in G$. The first one is the trivial method, that is calculating
		$g^1, g^2, ... g^k$ until $g^k = 1$, which is unfeasable if ord(G) is large. \\
		The second one requires us to know (or easily find) the prime decomposition of ord(G), which we can assume we don't, and thus the 
		assumption that finding ord(g) is hard is reasonable.

		So based on this assumption the above proof shows that the hash function is collision resistant.

\end{document}
