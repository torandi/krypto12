\documentclass[a4paper,11pt]{article}

\usepackage{listings}
\usepackage[T1]{fontenc}
\usepackage[english, english]{babel}
\usepackage{parskip}

\usepackage{rapportfram}

\author{Andreas Tarandi\\890416-0317}
\title{Homework D}
\titleextra{Study group: }
\course{DD2448}

\begin{document}
	\maketitle

	\section*{1 }
		To be collision resistant the following must hold:\\
		For all polynomial time adveraries A:\\
		$P[A(g, N) = (x, x') \wedge x \neq x' \wedge g^x mod N = g^{x'} mod N]$ is negligible\\
		That is, there exist no A that can find $x$ and $x'$ that are not equal for which\\
		$g^x \equiv g^{x'} mod N$\\
		which gives $x' = x + ord(g)k$, $k \in Z$.\\
		So finding a collision becomes as hard as finding the order of g, so if we can make the assumption that finding ord(g) of an element $g \in Z_{pq}$ where p and q are safe primes then this proof would be done.

		There are basicaly two ways of finding the order of an element $g \in G$. The first one is the trivial method, that is calculating
		$g^1, g^2, ... g^k$ until $g^k = 1$, which is unfeasable if ord(G) is large. \\
		The second one requires us to know (or easily find) the prime decomposition of ord(G), which we can assume we don't, and thus the 
		assumption that finding ord(g) is hard is reasonable.

		So based on this assumption the above proof shows that the hash function is collision resistant.

	\section*{2}
		For this we need to prove that
		$P[A(K) = (M, M') \wedge M \neq M' \wedge f_K(M) = f_K(M')$ is negligible.\\
		To do this I will consider the adversaries that can possibly generate the pair $(M, M')$. The first alternative is as in 1), to
		add a multiple of the order of $g_i$ to $m_i$, but as proved in 1) this is infeasable. 
		The other alternative is as follows:\\
		For each $m_i \in M$ the hash function $f_K$ generates a value $g_i^{m_i}$. Each of these value are then multiplied to get the
		final hash value, so if we could find $(m'_i, m'_j), i \neq j$ such that $m'_i = log_{g_i}(g_j^{m_j}), m'_j = log_{g_j}(g_i^{m_i})$ 
		we would get the same final hash. \\
		($g_i^{m_i} = g_j^{log_{g_j}(g_i^{m_i})}$)\\

		A variant of the second alternative would be to move factors of one of the factors to another (ie $a = g_i^{m_i} = cd, b = g_j^{m_j}$ and
		change $m_i$ and $m_j$ such that $a = c$ and $b = b*d$, but to do that one would have to calculate $log_{g_i}(d)$ and $log_{g_j}(d)$ both.

		Thus alternative one is unfeasable, and both variants of alternative two depends on the ability to calculate the discrete logarithm
		in $G_q$, but that was hard, so thus the hash is collision resistant.

	\section*{3 }
		\subsection*{3a }
			For polynomials in $Z_2$ there exists possible linear factors: $(x-0)$ and $(x-1)$.

			\subsubsection*{$x$}
				Irreducible.
			\subsubsection*{$x^5 + x^2 + x + 1$}
				Reducible to $(x - 1)*(x^4 + x^3 + x^2 + 1)$
			\subsubsection*{$x^3 + 1$}
				Reducible to $(x - 1)(x^2 + x + 1)$
			\subsubsection*{$x^7 + x^6 + 1$}
				Irreducible.\\
				Does not contain any linear- or quadratic factors. To be reducible it must contain factors such that:\\
				$x^7 + x^6 + 1 = ((Ax^4 + Bx^3 + Cx^2 + Dx + E) * (Fx^3 + Gx^2 + Hx + J)$\\
				By expansion this gives us:\\
				(1) $AF = 1$\\
				(2) $BF + AG = 1$\\
				(3) $CF + BG + AH = 0$\\
				(4) $DF + CG + BH + AJ = 0$\\
				(5) $EF + DG + CH + BJ = 0$\\
				(6) $EG + DH + CJ = 0$\\
				(7) $EH + DJ = 0$\\
				(8) $EJ = 1$\\
		
				(1) and (8) gives us $A = F = E = J = 1$\\

				This transforms the expressions (2)-(7) to:\\
				(Results from the preceding calculations are used)\\

				(2)	$B + G = 1$\\
						$B = G + 1$\\
				(3)	$C + BG + H = 0$\\
						$C + (G+1)G = H $\\
						$C = H + G^2 + G = H + G+G = H$\\
						$C = H$\\
				(4)	$D + CG + (G+1)H + 1 = 0$\\
						$D + HG + GH+H = 1$\\
						$D + H = 1$\\
						$D = H + 1$\\
				(5)	$1 + DG + CH + B = 0$\\
						$(H + 1)G + H^2 + G + 1 = 1$\\
						$HG + G + H + G = 0$\\
						$HG + H = 0$\\
						$GH = H$\\
				(6)	$G + DH + C = 0$\\
						$G + (H+1)H + H = 0$\\
						$G + H + H + H = 0$\\
						$G + H = 0$\\
						$G = H$\\
				(7)	$H + D = 0$\\
						$H = D$\\

				But then (7) and (4) contradicts each other, and this can not hold, thus the
				polynomial is irreducible.

		\subsection*{3b}
			All calculations are in $Z_2[x]$\\
			\subsubsection{$(x^5+x+1)(x^3+x^2+1)$}
				$(x^5+x+1)(x^3+x^2+1) = x^8 + x^7 + x^5 + x^4 + x^2 + x + 1$
				Reducing this by $x^6 + x + 1$ gives the remainder $x^5 + x^4 + x^3 + x + 1$ which is also the result of the calculation.

				\subsubsection{ }

			
\end{document}
